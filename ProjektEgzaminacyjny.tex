\documentclass[a4paper,11pt]{article}

\usepackage{amsfonts}
\usepackage{amsmath}
\usepackage{amssymb}
\usepackage{graphicx}

\usepackage[utf8]{inputenc}
%\usepackage[cp1250]{inputenc}
\usepackage[polish]{babel}
\usepackage[T1]{fontenc}

%----------------------
\def\\{\hfill\break}


%----------------------
\title{Projekt Egzaminacyjny}
\author{nazwiska}
\date{Listopad 2021 -- Styczeń 2022}

\begin{document}

\maketitle

\section{Wstęp}
Jeśli to jest Twoja pierwsza praca w LaTex, pomocna może być strona:
\smallskip

https://www.overleaf.com/learn/latex/Learn\_LaTeX\_in\_30\_minutes

\section {Analiza cen spółek}
W paragrafach 1 i 2 w kilku zdaniach opisz każdą ze spółek. Zredaguj krótko,
precyzyjnie wyniki. Zamieść wykorzystaną teorię.

\subsection{Spółka 1}
Przykładowy sposób zamieszczenia wykresu

\centerline{\includegraphics[width=6cm]{QQplot1.png}}

\begin{minipage}[c]{6cm}
Dodatkowo tak można organizować przestrzeń pracy.
\end{minipage}\hfill
\begin{minipage}[c]{5cm}
\includegraphics[width=5cm]{QQplot1.png}
\end{minipage}

\\Symbole matematyczne piszemy w dolarach \$x\$ i wtedy otrzymamy:   $x$.
Ważne wzory często chcemy wycentrować, wtedy zamiast jednego dolara, dajemy dwa
\$\$x+y=y+x\$\$ i otrzymamy:
$$x+y=y+x$$

\subsection{Spółka 2}


\section{Rozdział 2}
\section{Rozdział 3}
\section{Podsumowanie}
\end{document}

